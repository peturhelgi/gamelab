\section{Final Results}
Our final version of the game is a two player 2D platformer with an advanced level editor. The two players must collaboratively solve puzzles to escape from a collapsed mine.

The first level of the game contains a short story and a tutorial, in which the players learn the controls and the basic game mechanics. In-game, the players can mine rocks, throw ropes, use platforms with buttons and/or levers, pick up and drop crates, climb ropes and ladders and collect a key to unlock the exit door.
But the fun does not stop when all the included levels have been completed: by using the level editor, the player can create his own levels. He can switch back and forth between the level editor and the playable game to directly test the new creation. When he is finished, he can save the level and challenge others to solve his very own puzzles.
\newline\newline
\textbf{Improvements after the alpha release}

We made some changes after the alpha, most of which came up during the play-testing. The biggest change is how the tools work. Now players have only one life and a limited resource of tools, urging the player to think about their usage of resource.
Other improvements were made regarding the camera, so it doesn't zoom in too close or move around too much when the miners are close to each other. Further improvements were made to the rope, so that one must grab onto it and there is also a dim contour of the rope shown, in order for the player to know when they can actually place it. We further polished the level editor by adding saving functionality, platforms, buttons and levers. We added an introduction, a tutorial level and a challenging final level as well as some various minor changes to improve the gameplay. Furthermore, we fixed some intrusive bugs that were discovered in the play-testing and credits after the last level.

\section{Experience}
Since the initial idea, our game did not have any fundamental changes. During the journey of creating the game we stumbled across many ideas, which made the game more unique. For example, the limited vision which we discovered during the prototype testing to be a real \enquote{game changing} mechanism, changed the flair of the game a lot. Within the team, the vision of the final game was always similar.

We did not pay a lot of attention to the development schedule due to problems and lacking progress in some tasks, but the layered breakdown was very helpful to prioritize tasks and to estimate the current state of the project.
The play-testing was a good opportunity to get new, unbiased impressions on the game and was crucial to fix some things, which seemed intuitive for us, but were not for the testers. Inputs from experienced people (teaching assistants, mentor, ZhdK professors and students) were extremely valuable and helped us, to reflect on some game mechanics.

The biggest problem in our team was the lack of communication. In the beginning, the team worked on different versions of the game, leading to the difficult situation of merging the progress. After many days of unsuccessfully trying to do so, we decided on one version to continue with. Further we came up with a set of rules, which worked out well for some time. When further conflicts occurred, leading to a divergence within the team, we distributed the tasks to be as independent from each other as possible. This allowed us to efficiently use the remaining time and bring the project to a successful end.

\section{Personal Impressions}
\textit{Simon:}

The final game does meet most of my expectations. I would have especially liked to further invest time into the level editor and the lighting/rendering. Regarding the milestones, we managed to reach all goals up to the desired target and even expanded into some of the high target. Thinking about the course, it might have been interesting to hear some things about often used game patterns and maybe a short section about collaboration in a team, as participants might have very different experience in that field.
\newline\newline
\textit{Bjarni:}

I would have liked to add more features to the game but no matter what, I would have always felt that way and I am satisfied with what we accomplished. I do believe the game exceeded my expectations for it is my first and especially considering the time we had. The duration of the class is short when you think about the fact that we are supposed to develop a game in that time. But it is also a school course and that is a part of it, so the expectation towards the game should be reasonable as the project suddenly ends and the game might not be exactly where you wanted to leave it.

This years theme was different. It was difficult in the beginning to really decide on a game that fit the theme and the theme ended up being more of a back-story. The same goes for the other games this year. We don't look at the game and see the theme 'Alfred Escher' shine through as it would e.g. with a jungle theme. However, we don't believe total freedom is better, as some limitations will have a positive effect on creativity.
\newline\newline
\textit{Andreea:}

The final result of our game did meet my expectations when I think about the starting point, even though I would have liked to make it more interesting to the player, overall. For example, I would have liked to invest more into creating a story that would pique the interest of the player (however, now that I think about it, some people don’t give that much importance to the back-story of a game when playing). 

I am indeed proud of our game and of the entire evolution of the project, given the short amount of time we had at our disposal. Also, I am happy to have learned so many things while working on this game, e.g. improved my experience with C\#, created my first animations, visual assets, etc. 

Regarding the biggest technical difficulty we encountered in the project, I think that would be the fundamental code skeleton of our game, in the sense that we lacked such a thing in the beginning, which lead to various divergences in our team and plenty of time wasted. If we have had something on which everyone could agree from the start, I think the evolution of our game would have progressed differently. 
\newline\newline
\textit{Nicolas:}

Seeing the end result felt very satisfying, even if I felt that there are lots of features that can be added or improved upon. In my experience, this corresponds to the norm in every software project, thus, as a whole, I am very proud of what we accomplished. The tight time constraints imposed definitely posed some challenges, but I think that we learned a lot from having to deliver concrete milestones throughout the course.

Deciding on some baseline principles at the beginning (what kind of game, single/multiplayer, co-op/pvp etc.) proved to be a tremendous help. If I were to do the project again, I would spend even more time drafting these baseline principles such as to avoid conflicts or disappointments later on.
\newline\newline
\textit{Petur:}

The course in a whole met and exceeded my expectations. However, I did not believe in the game idea so I can’t say I’m very proud of the game. Although it was fun and challenging to work on it, it would have been better to work on a game I think that can be fun. The schedule was a bit tight, but given the scope of the course, I think it was manageable and kept us busy. The theme was quite challenging but lent itself to a broad spectrum of games. I have mixed feelings about using MonoGame for this kind of project. I felt that the provided flow was nice and good to work with, but the MonoGame pipeline tool was a bit limiting. It felt a bit awkward to use a shader that is loaded through the pipeline. Documentation is also not splendid and most help was acquired through forums.

In retrospect, I would say the project was a success, but there are many things that can be added and improved. We managed to make it a playable game, but it needs more levels, more tools and more interactable elements in order to make it interesting. 

In my next game project, I would spend more time on software engineering, so that patterns, skeletons and protocols are created before team members start working individually. I would also constantly criticize the working structure, so that it can be changed if proven suboptimal. There is also a need for a better git workflow, as the one we used was not the best. Core functionality should be established before coding begins and should always be the main goal.

\subsection*{Final look of the game}

\begin{figure}[h!]
    \centering
    \includegraphics[width=\textwidth]{img/conclusion/menu_image.png}
    \caption{The in-game main menu}
\end{figure}

\begin{figure}[h!]
    \centering
    \includegraphics[width=\textwidth]{img/conclusion/ingame.png}
    \caption{A screenshot of actual gameplay}
\end{figure}

\begin{figure}[h!]
    \centering
    \includegraphics[width=\textwidth]{img/conclusion/level_editor.png}
    \caption{The picker wheel that allows the player to place objects 
    in an entirely new level}
\end{figure}


